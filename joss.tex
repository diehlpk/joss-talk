%------------------------------------------------
\section{The Journal of Open Source Software: Developing a Software Review Community} % Sections can be created in order to organize your presentation into discrete blocks, all sections and subsections are automatically printed in the table of contents as an overview of the talk
%------------------------------------------------

%\subsection{Subsection Example} % A subsection can be created just before a set of slides with a common theme to further break down your presentation into chunks

\begin{frame}{Journal of Open Source Software}
\framesubtitle{Developing a Software Review Community}
\begin{center}
\includegraphics{joss.png}    
\end{center}

{\tiny
Gabriela Alessio Robles, Mikkel Meyer Andersen, Katy Barnhart, Juanjo Bazán, Sebastian Benthall, Eloisa Bentivegna,
Monica Bobra, Frederick Boehm, Jed Brown, Pierre de Buyl, Patrick Diehl, Elizabeth DuPre, Vissarion Fisikopoulos,
Martin Fleischmann, Dan Foreman-Mackey, Jarvis Moore Frost, Nikoleta Glynatsi, Jeff Gostick, Richard Gowers, Hugo Gruson,
Olivia Guest, David Hagan, Jayaram Harihan, Chris Hartgerink, Bita Hasheminezhad, Christina Hedges, Luiz Irber,
Mark A. Jensen, Prashant K. Jha, \textcolor{blue}{Daniel S. Katz}, Vincent Knight, Rachel Kurchin, Hugo Ledoux, Christopher R. Madan,
Brian McFee, Melissa Weber Mendonça, Kevin M. Moerman, Kyle Niemeyer, Juan Nunez-Iglesias, Lorena Pantano,
Stefan Pfenninger, Viviane Pons, Kristina Riemer, Amy Roberts, Marie E. Rognes, Ariel Rokem, Will Rowe, Kelly Rowland,
David P. Sanders, Mehmet Hakan Satman, Fabian Scheipl, Jacob Schrieber, Adi Singh, \textcolor{blue}{Arfon Smith}, Charlotte Soneson,
Øystein Sørensen, Andrew Stewart, Fabian-Robert Stöter, Yuan Tang, George K. Thiruvathukal, Kristen Thyng,
Tim Tröndle, Leonardo Uieda, Chris Vernon, Marcos Vital, Lucy Whalley, Bruce E. Wilson, Frauke Wiese}

\begin{center}
     \url{https://joss.theoj.org}
\end{center}

\end{frame}

\begin{frame}{How to better recognize software
contributions?}

\begin{itemize}
    \item Find some way to fit software into
current (paper/book-centric)
system
    \item  Evolve beyond one-dimensional
credit model
\end{itemize}

\begin{center}
    \textcolor{blue}{What if we just wrote papers about software?}
\end{center}

\begin{itemize}
    \item<only@1> Gives us something easy to cite \emoji{thumbs-up-light-skin-tone}
    \item<only@1> No changes required to existing
infrastructure \emoji{ok-hand-medium-dark-skin-tone}
    \item<only@1> Publishing in existing journals raises
profile of software within a community \emoji{love-you-gesture-medium-skin-tone}
%
    \item<2> Writing another paper can be a ton of
work \emoji{grinning-face-with-sweat}
    \item<2> Many journals don’t accept software
papers \emoji{face-with-symbols-on-mouth}
    \item<2> For long-lived software packages, static
authorship presents major issues \emoji{worried-face}
    \item<2> Many papers about the same software
may lead to citation dilution \emoji{oncoming-fist-light-skin-tone}
\end{itemize}

    
\end{frame}

\begin{frame}

\begin{center}
What if we made it as easy as
possible to write and publish a
software paper?
\end{center}

\begin{center}
   \textcolor{blue}{Embracing the hack}
\end{center}

\end{frame}

\begin{frame}
  \includegraphics[scale=0.5]{joss.png}      \includesvg{joss-text.svg}
\vspace{1cm}
\begin{itemize}
    \item A \textbf{developer-friendly} journal* for research software
packages
    \item Paper preparation (and submission) for well-documented
software should take \textbf{no more than an hour}
\item The primary purpose of a JOSS paper is to \textbf{enable citation
credit} to be given to authors of research software
\end{itemize}

\vspace{2cm}

{tiny * Other (commercial) venues exist for publishing papers about software}
\end{frame}

    
\begin{frame}{JOSS process {\tiny(\url{https://doi.org/10.6084/m9.figshare.5147773.v2})}}
    \centering
    \includegraphics[scale=0.5]{JOSS-flowchart-updated.pdf}
\end{frame}

\begin{frame}{JOSS - Review Checklist}

\begin{itemize}
    \item Agree to Conflict of Interest \& Code of Conduct
    \item General checks: repository URL, license,
contribution and authorship
    \item Functionality: installation, functional claims,
performance
\item Documentation: statement of need, installation
instructions, example usage, functionality
documentation, automated tests, community
guidelines
\item Software paper: summary, statement of need,
state of the field, quality of writing, references
\end{itemize}
    
\end{frame}


\begin{frame}{JOSS - Review Checklist Details}
Definition of each check-in JOSS documentation:
\url{https://joss.readthedocs.io/en/latest/
review_criteria.html}\\
\vspace{0.5cm}
The editor helps the reviewer and author come to
agreement and some criteria have guidance
\vspace{0.5cm}
\begin{itemize}
    \item Installation
    \item API documentation
    \item Community guidelines
    \item Automated testing
\end{itemize}
    
\end{frame}

\begin{frame}{JOSS - Review Checklist Details}
\framesubtitle{Installation}

\begin{itemize}
    \item \emoji{thumbs-up-medium-skin-tone}: The software is simple to install, and
follows established distribution and dependency
management approaches for the language
being used
\item \emoji{oncoming-fist-dark-skin-tone}: A list of dependencies to install, together
with some kind of script to handle their
installation (\emph{e.g.}, a Makefile)
\item \emoji{thumbs-down-light-skin-tone} (not acceptable): Dependencies are unclear,
and/or installation process lacks automation
\end{itemize}
    
\end{frame}

\begin{frame}{JOSS - Review Checklist Details}
\framesubtitle{API Documentation}

\begin{itemize}
        \item \emoji{thumbs-up-medium-skin-tone}: All functions/methods are documented
including example inputs and outputs
\item \emoji{oncoming-fist-dark-skin-tone}: Core API functionality is documented
\item \emoji{thumbs-down-light-skin-tone} (not acceptable): API is undocumented
\end{itemize}
    
\end{frame}

\begin{frame}{JOSS - Review Checklist Details}
\framesubtitle{Automated Tests}

\begin{itemize}
        \item \emoji{thumbs-up-medium-skin-tone}: An automated test suite hooked up to
continuous integration (GitHub Actions, Circle CI,
or similar)
\item \emoji{oncoming-fist-dark-skin-tone}: Documented manual steps that can be
followed to objectively check the expected
functionality of the software (\emph{e.g.}, a sample input
file to assert behavior)
\item \emoji{thumbs-down-light-skin-tone} (not acceptable):  No way for you, the
reviewer, to objectively assess whether the
software works
\end{itemize}
    
\end{frame}

\begin{frame}{JOSS as a Community}
    \begin{itemize}
        \item Cultures change based on rules and incentives
        \item JOSS practices have influenced reviewers and
developers in terms of what's good and what's
minimally acceptable
\item Similar to rOpenSci's influence in the R community
\item JOSS provides rules, and at a high-level, tries to
nudge incentives
\item Accepted software = accepted paper
\item If software was cited directly, JOSS papers wouldn't
be needed, but JOSS reviews and JOSS
community would still have great value
    \end{itemize}
\end{frame}


\begin{frame}\frametitle{Bot-based process using GitHub: @editorialbot}

    \begin{columns}
    \column{0.5\textwidth}
    \begin{itemize}
        \item Interacts with authors, reviewers, and editors
in review ‘issues’ on GitHub
\item Compiles papers (Pandoc)
\item Conducts automated ‘healthchecks’ for
incoming submissions (\emph{e.g.}\ license checks,
search for missing DOIs)
\item Sends automated reminders
\item Deposits metadata and
 and 
registers DOIs with Crossref
    \end{itemize}
        
    \column{0.6\textwidth}
        \centering
        \includegraphics[width=\textwidth]{joss-github.png}
    \end{columns}

\end{frame}

\begin{frame}{Some data on most cited papers}

\includegraphics[width=\linewidth]{most-cited.png}



{\tiny More data: \url{http://www.theoj.org/joss-analytics/joss-submission-analytics.html}}    
\end{frame}

\begin{frame}{Some data on published papers}

\includegraphics[width=\linewidth]{papers-per-year.png}



{\tiny More data: \url{http://www.theoj.org/joss-analytics/joss-submission-analytics.html}}    
\end{frame}


\begin{frame}{Conclusion}

\begin{center}
\Large \textcolor{blue}{JOSS is a collaboration
between author, editor and
reviewer}
\end{center}
\vspace{1cm}
Please submit your software package and get credit for your work!
\vspace{1cm}
{\tiny Slides based on: \url{https://doi.org/10.5281/zenodo.6305241}\\
Guest Post — The Evolving Role of Scientific Editing: \url{https://scholarlykitchen.sspnet.org/2021/09/23/guest-post-the-evolving-role-of-scientific-editing/}}
    
\end{frame}