%------------------------------------------------
%\section{The Journal of Open Source Software: Developing a Software Review Community} % Sections can be created in order to organize your presentation into discrete blocks, all sections and subsections are automatically printed in the table of contents as an overview of the talk
%------------------------------------------------

%\subsection{Subsection Example} % A subsection can be created just before a set of slides with a common theme to further break down your presentation into chunks

\begin{frame}{Journal of Open Source Software}
\begin{center}
\includegraphics{joss.png}    
\end{center}

{\fontsize{4}{8}\selectfont
\textbf{Patrick Diehl}, Daniel S.~Katz, Gabriela Alessio Robles, Stefan Appelhoff, Warrick Ball,  Mojtaba Barzegari, Johanna Bayer,  Juanjo Bazán,  Sophie Beck,  Sebastian Benthall, Eloisa Bentivegna,  Monica Bobra,  Frederick Boehm, Sébastien Boisgérault, Josh Borrow, Teon Brooks, Jed Brown, Philip Cardiff, Taher Chegini,  Beatriz Costa Gomes, Pierre de Buyl, Renata Diaz,  Axel Donath, Elizabeth DuPre, Matthew Feickert, Vissarion Fisikopoulos, Martin Fleischmann, Samuel Forbes, Dan Foreman-Mackey, Jarvist Moore Frost, Nikoleta Glynatsi, Jeff Gostick, Rohit Goswami, Richard Gowers, Hugo Gruson, Olivia Guest, Jayaram Hariharan, Gracielle Higino, Susan Holmes, Luiz Irber, Adam R. Jensen, Mark A. Jensen, Prashant K Jha, Sehrish Kanwal, Vincent Knight, Olexandr Konovalov, Rachel Kurchin, Paul La Plante, Oskar Laverny, Hugo Ledoux, Christopher R. Madan, Michael Mahoney, Brian McFee, Rocco Meli, Sarath Menon, Antonia Mey, Tristan Miller, Kevin M. Moerman, Ivelina Momcheva, Yasmin Mzayek, Kanishka B. Narayan, Kyle Niemeyer, Lorena Pantano, Andrew Quinn, AHM Mahfuzur Rahman, Julia Romanowska, Kelly Rowland, Anjali Sandip, Mehmet Hakan Satman, Jonny Saunders, Fabian Scheipl, Jacob Schreiber, Hauke Schulz, Adi Singh, Arfon Smith, Dana Solav, Claudia Solis-Lemus, Charlotte Soneson, Øystein Sørensen, Andrew Stewart, Marcel Stimberg, Fabian-Robert Stöter, Fei Tao, George K. Thiruvathukal, Kristen Thyng, Ana Trisovic, Adam Tyson, Chris Vernon, Marcos Vital, Rachel Wegener, Britta Westner, Lucy Whalley, Frauke Wiese, Mengqi Zhao, and Bonan Zhu.}

\begin{center}
     \url{https://joss.theoj.org}
\end{center}

\end{frame}

\section{How to better recognize software
contributions?}

\begin{frame}{How to better recognize software
contributions?}

\begin{itemize}
    \item Find some way to fit software into
current (paper/book-centric)
system
    \item  Evolve beyond one-dimensional
credit model
\end{itemize}

\begin{center}
    \textcolor{azure}{What if we just wrote papers about software?}
\end{center}

\begin{itemize}
    \item<only@1> Gives us something easy to cite \emoji{thumbs-up-light-skin-tone}
    \item<only@1> No changes required to existing
infrastructure \emoji{ok-hand-medium-dark-skin-tone}
    \item<only@1> Publishing in existing journals raises
profile of software within a community \emoji{love-you-gesture-medium-skin-tone}
%
    \item<2> Writing another paper can be a ton of
work \emoji{grinning-face-with-sweat}
    \item<2> Many journals don’t accept software
papers \emoji{face-with-symbols-on-mouth}
    \item<2> For long-lived software packages, static
authorship presents major issues \emoji{worried-face}
    \item<2> Many papers about the same software
may lead to citation dilution \emoji{oncoming-fist-light-skin-tone}
\end{itemize}

    
\end{frame}

\begin{frame}

\begin{center}
What if we made it as easy as
possible to write and publish a
software paper?
\end{center}

\begin{center}
   \textcolor{azure}{Embracing the hack}
\end{center}

\end{frame}

\section{Submission process}

\begin{frame}
  \includegraphics[scale=0.5]{joss.png}      \includesvg{joss-text.svg}
\vspace{1cm}
\begin{itemize}
    \item A \textbf{developer-friendly} journal* for research software
packages
    \item Paper preparation (and submission) for well-documented
software should take \textbf{no more than an hour}
\item The primary purpose of a JOSS paper is to \textbf{enable citation
credit} to be given to authors of research software
\end{itemize}

\vspace{2cm}

{tiny * Other (commercial) venues exist for publishing papers about software}
\end{frame}

    
\begin{frame}{JOSS process {\tiny(\url{https://doi.org/10.6084/m9.figshare.5147773.v2})}}
    \centering
    \includegraphics[scale=0.5]{JOSS-flowchart-updated.pdf}
\end{frame}

\begin{frame}{JOSS - Review Checklist}

\begin{itemize}
    \item Agree to Conflict of Interest \& Code of Conduct
    \item General checks: repository URL, license,
contribution and authorship
    \item Functionality: installation, functional claims,
performance
\item Documentation: statement of need, installation
instructions, example usage, functionality
documentation, automated tests, community
guidelines
\item Software paper: summary, statement of need,
state of the field, quality of writing, references
\end{itemize}
    
\end{frame}


\begin{frame}{JOSS - Review Checklist Details}
Definition of each check-in JOSS documentation:
\url{https://joss.readthedocs.io/en/latest/
review_criteria.html}\\
\vspace{0.5cm}
The editor helps the reviewer and author come to
agreement and some criteria have guidance
\vspace{0.5cm}
\begin{itemize}
    \item Installation
    \item API documentation
    \item Community guidelines
    \item Automated testing
\end{itemize}
    
\end{frame}

\begin{frame}{JOSS - Review Checklist Details}
\framesubtitle{Installation}

\begin{itemize}
    \item \emoji{thumbs-up-medium-skin-tone}: The software is simple to install, and
follows established distribution and dependency
management approaches for the language
being used
\item \emoji{oncoming-fist-dark-skin-tone}: A list of dependencies to install, together
with some kind of script to handle their
installation (\emph{e.g.}, a Makefile)
\item \emoji{thumbs-down-light-skin-tone} (not acceptable): Dependencies are unclear,
and/or installation process lacks automation
\end{itemize}
    
\end{frame}

\begin{frame}{JOSS - Review Checklist Details}
\framesubtitle{API Documentation}

\begin{itemize}
        \item \emoji{thumbs-up-medium-skin-tone}: All functions/methods are documented
including example inputs and outputs
\item \emoji{oncoming-fist-dark-skin-tone}: Core API functionality is documented
\item \emoji{thumbs-down-light-skin-tone} (not acceptable): API is undocumented
\end{itemize}
    
\end{frame}

\begin{frame}{JOSS - Review Checklist Details}
\framesubtitle{Automated Tests}

\begin{itemize}
        \item \emoji{thumbs-up-medium-skin-tone}: An automated test suite hooked up to
continuous integration (GitHub Actions, Circle CI,
or similar)
\item \emoji{oncoming-fist-dark-skin-tone}: Documented manual steps that can be
followed to objectively check the expected
functionality of the software (\emph{e.g.}, a sample input
file to assert behavior)
\item \emoji{thumbs-down-light-skin-tone} (not acceptable):  No way for you, the
reviewer, to objectively assess whether the
software works
\end{itemize}
    
\end{frame}

\begin{frame}{JOSS as a Community}
    \begin{itemize}
        \item Cultures change based on rules and incentives
        \item JOSS practices have influenced reviewers and
developers in terms of what's good and what's
minimally acceptable
\item Similar to rOpenSci's influence in the R community
\item JOSS provides rules, and at a high-level, tries to
nudge incentives
\item Accepted software = accepted paper
\item If software was cited directly, JOSS papers wouldn't
be needed, but JOSS reviews and JOSS
community would still have great value
    \end{itemize}
\end{frame}

\section{Editorial bot}

\begin{frame}\frametitle{Bot-based process using GitHub: @editorialbot}

    \begin{columns}
    \column{0.5\textwidth}
    \begin{itemize}
        \item Interacts with authors, reviewers, and editors
in review ‘issues’ on GitHub
\item Compiles papers (Pandoc)
\item Conducts automated ‘healthchecks’ for
incoming submissions (\emph{e.g.}\ license checks,
search for missing DOIs)
\item Sends automated reminders
\item Deposits metadata and
 and 
registers DOIs with Crossref
    \end{itemize}
        
    \column{0.6\textwidth}
        \centering
        \includegraphics[width=\textwidth]{joss-github.png}
    \end{columns}

\end{frame}

\section{9 years of JOSS in numbers}

\begin{frame}{Read more about JOSS and where the data is from}

\begin{itemize}
\item Diehl et al (2024): The Journal of Open Source Software (JOSS) - Bringing open source software practices to the scholarly publishing community for authors, reviewers, editors, and publishers, Journal of Librarianship and Scholarly Communication (JLSC), submitted.
\item JOSS Submission Analytics\footnote{\tiny\url{http://www.theoj.org/joss-analytics/joss-submission-analytics.html}}. Kudos to Charlotte Soneson.
\end{itemize}
\begin{block}{Questions to ask?}
\begin{itemize}
\item Are JOSS papers cited?
\item Is JOSS growing?
\item How many people volunteer?
\item Why do people publish in JOSS?
\end{itemize}
\end{block}

\end{frame}


\begin{frame}{Are JOSS papers cited?}

\includegraphics[width=\linewidth]{most-cited.png}

{\tiny More data: \url{http://www.theoj.org/joss-analytics/joss-submission-analytics.html}}    
\end{frame}

\begin{frame}{Some data on published papers}

\includegraphics[width=\linewidth]{papers-per-year.png}

Figure \textbf{A}. The number of papers published by JOSS each month since the inception in 2016. The smooth curve represents a LOESS fit to the monthly data. \textbf{B}. The number of distinct editors accepting at least one submission in a given year.

{\tiny More data: \url{http://www.theoj.org/joss-analytics/joss-submission-analytics.html}}    
\end{frame}

\begin{frame}{Some data on reviews and reviewers}

\includegraphics[width=\linewidth]{reviews-and-reviewers}

\small Figure \textbf{A}. Distribution of the number of reviewers assigned to a JOSS submission. All submissions since 2020 were evaluated by at least two reviewers (with the exception of two published addenda, and noting that in three cases, the second review was completed by the editor after the reviewer dropped out of the review process) \textbf{B}. The number of reviewers evaluating at least one JOSS submission in a given year, as well as the total number of reviews performed that year. 


{\tiny More data: \url{http://www.theoj.org/joss-analytics/joss-submission-analytics.html}}    
\end{frame}

\begin{frame}{Some data on the details of comments and length}

\includegraphics[width=\linewidth]{comments-and-lengths}

\small Figure: \textbf{A}. Distribution of the number of comments made in the review issue for JOSS submissions (note that reviewers typically also open issues in the repository of the software being reviewed, but these are not captured in this figure). Comments made by the JOSS editorial bot are excluded. \textbf{B}. The total time spent in pre-review and review states.


{\tiny More data: \url{http://www.theoj.org/joss-analytics/joss-submission-analytics.html}}    
\end{frame}

\begin{frame}{Why do people publish in JOSS}
\vspace{-0.5cm}
\begin{center}
\includegraphics[width=0.8\linewidth]{joss-cast}
\end{center}

\tiny Figure: Number of JOSSCast episodes (out of 11 total considered) in which authors, when asked about their choice of JOSS or experience with JOSS, mentioned each of six key ideas. Papers analyzed span six of the eight topical tracks and were published between March 2023 and April 2024 and had a median length of time in review of 70 days, making them relatively statistically representative of recent publications.

{\tiny JossCast: \url{https://blog.joss.theoj.org/}}    
\end{frame}

\section{Costs per JOSS paper? Disclaimer: JOSS has no publication fee!}

\begin{frame}{ \emoji{heavy-dollar-sign} How much does it cost to publish one paper?}
\framesubtitle{JOSS has no publication or open access fees! \emoji{clapping-hands}}

\begin{block}{Costs in 2019\footnote{\tiny\url{https://blog.joss.theoj.org/2019/06/cost-models-for-running-an-online-open-journal}}}
\begin{itemize}
\item  Annual Crossref membership: \$275/year
\item  JOSS paper DOIs: \$1/accepted paper
\item  JOSS website hosting: \$19/month
\item  JOSS domain name registration: \$10/year
\end{itemize}
\end{block}
At 300 papers/year in 2019, this is \$813, or \textcolor{azure}{\$2.71/paper}. This is what we actually pay today, covered by a grant from the Alfred P. Sloan Foundation.\\
\vspace{0.5cm}
Today, a more accurate cost is probably around \textcolor{azure}{\$4-US\$5 per paper}, though this hasn't been formally calculated, including services such as web hosting, Crossref membership and services, Portico preservation services, etc. 

\end{frame}

\begin{frame}{No publication fees? Who finances JOSS? \emoji{detective}}

\begin{itemize}
\item \$50 for each paper that comes from AAS (24 such papers have been published by JOSS)
\item \$20k gift from the Gordon and Betty Moore Foundation to start up the journal infrastructure 
\item \$380k grant from Alfred P. Sloan Foundation to improve the infrastructure and generalize it to be useful to other parts of the scholarly publishing community. 
\item Some authors donated to JOSS after the paper was accepted.
\end{itemize}

\end{frame}

